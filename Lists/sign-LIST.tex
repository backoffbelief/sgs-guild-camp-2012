%%%%
%%
%% This file sets up the Sign and Label datatypes and creates Sign and
%% Label macros.
%%
%% Signs generally represent interesting parts of game area, usually
%% as things posted on walls.  Labels represent other things, often on
%% or inside envelopes, that are part of complex mechanics.
%%
%% The default value for \MYloc will inherit location from the Place
%% or Sign most immediately up the ownership tree.  Override this by
%% setting \MYloc to anything (even blank).
%%
%% Sign is for full-sized signs that would cover most of a large
%% manila envelope; SignMedium is for signs sized to half-sized manila
%% envelopes; SignSmall is for signs sized for small manila envelopes
%% (the same size as item cards).  Label, LabelMedium, and LabelSmall
%% are analagous, but they don't have a \takedownby note at the
%% bottom.  You can always use a sign or label without an envelope or
%% with a differently-sized envelope.  Choose which based on
%% visibility and content.
%%
%% SignTiny is for signs you want to be hard to find; it is small and
%% does not have a \takedownby note.  SignDot is for a very small
%% "dot" which only has a title.
%%
%% SignStrip produces a strip of paper (without a \takedownby note)
%% with labels on the outside that show on both sides if you fold it
%% in half.  These are a convenient alternative to sub-envelopes. They
%% can also be used for "s-packets" taped to walls (see
%% Extras/README-s-packets).
%%
%% LabelCover produces a label similar to the cover to a research
%% notebook.  LabelPage, likewise, produces a page.
%%
%% EOG is for full-sized end-of-game signs.
%%
%%%%%

\DECLARESUBTYPE{Sign}{Element}
\PRESETS{Sign}{
  \FD\MYloc	{\mylocation} %% real-space location
  \FD\MYtext	{} %% text of sign
  }
\POSTSETS{Sign}{
  \edef\mylocation{\MYloc}
  \protected@edef\@ownerstring{%
    \MYname%
    \ifx\mylocation\empty\else\ (\mylocation)\fi%
    }
  }
\def\mylocation{}

\def\loc#1{\rs\MYloc{#1}}

\DECLARESUBTYPE{SignMedium}{Sign}
\DECLARESUBTYPE{SignSmall}{Sign}
\DECLARESUBTYPE{SignTiny}{Sign}
\DECLARESUBTYPE{SignDot}{Sign}
\PRESETS{SignDot}{\s\MYtext{}}

\DECLARESUBTYPE{Label}{Sign}
\PRESETS{Label}{\s\MYloc{}}
\DECLARESUBTYPE{LabelMedium}{Label}
\DECLARESUBTYPE{LabelSmall}{Label}

\DECLARESUBTYPE{SignStrip}{Sign}
\DECLARESUBTYPE{LabelCover}{Label}
\DECLARESUBTYPE{LabelPage}{Label}

\DECLARESUBTYPE{EOG}{Sign}
\PRESETS{EOG}{%
  \s\MYname	{End Of Game}
  \s\MYtext	{{\bf\Huge You may not pass through here.}}
  }


%%%%%
%% \signbig[<location>]{<name>}{<text>}
%% \eog[<location>]
%%
%% \signmdeium[<location>]{<name>}{<text>}
%% \signsmall[<location>]{<name>}{<text>}
%% \signtiny[<location>]{<name>}{<text>}
%% \signdot[<location>]{<name>}
%%
%% \labelbig{<name>}{<text>}
%% \labelmedium{<name>}{<text>}
%% \labelsmall{<name>}{<text>}
%%
%% \signstrip[<location>]{<name>}{<text>}
%% \labelcover{<name>}{<text>}
%% \labelpage{<name>}{<text>}
\newinstance{Sign}{\signbig[3][\mylocation]}{
  \s\MYloc{#1}\s\MYname{#2}\s\MYtext{#3}}
\newinstance{EOG}{\eog[1][\mylocation]}{\s\MYloc{#1}}

\newinstance{SignMedium}{\signmedium[3][\mylocation]}{
  \s\MYloc{#1}\s\MYname{#2}\s\MYtext{#3}}
\newinstance{SignSmall}{\signsmall[3][\mylocation]}{
  \s\MYloc{#1}\s\MYname{#2}\s\MYtext{#3}}
\newinstance{SignTiny}{\signtiny[3][\mylocation]}{
  \s\MYloc{#1}\s\MYname{#2}\s\MYtext{#3}}
\newinstance{SignDot}{\signdot[2][\mylocation]}{
  \s\MYloc{#1}\s\MYname{#2}}

\newinstance{Label}{\labelbig[2]}{
  \s\MYname{#1}\s\MYtext{#2}}
\newinstance{LabelMedium}{\labelmedium[2]}{
  \s\MYname{#1}\s\MYtext{#2}}
\newinstance{LabelSmall}{\labelsmall[2]}{
  \s\MYname{#1}\s\MYtext{#2}}

\newinstance{SignStrip}{\signstrip[3][\mylocation]}{
  \s\MYloc{#1}\s\MYname{#2}\s\MYtext{#3}}
\newinstance{LabelCover}{\labelcover[2]}{
  \s\MYname{#1}\s\MYtext{#2}}
\newinstance{LabelPage}{\labelpage[2]}{
  \s\MYname{#1}\s\MYtext{#2}}


%%%%%
%% \sEOG{}
%% use \sEOg[\loc{<location>}]{} for EOG sign at a specific place
\NEW{EOG}{\sEOG}{
  }


%%%%%%%%%%%%%%%%%%%%%%%%%%%%%%%%%%%%%%%%%%%%%%%%%%%%%%%%%%%%%%%%%%

\NEW{Sign}{\sTest}{
  \s\MYname	{A Room}
  \s\MYloc	{10-250}
  \s\MYtext	{A lecture hall with large, sliding blackboards.}
  }


%%%%%%%%%%%%%%%%%%%%%%%%%%%%%%%%%%%%%%%%%%%%%%%%%%%%%%%%%%%%%%%%%%

%% The hotel rooms
\NEW{Sign}{\sBrideRoom}{
  \s\MYname	{\cBride{\MYname{}}'s Hotel Room}
  \s\MYloc	{- 004}
  \s\MYtext	{The nameplaque says \cBride{\MYname{}}. If you have the room key with badge number 100, you may enter the room.}
  \s\MYitems {\iBible}
  }
  
\NEW{Sign}{\sGroomRoom}{
  \s\MYname	{\cGroomA{\MYname{}}'s Hotel Room}
  \s\MYloc	{- 004}
  \s\MYtext	{The name plaque says \cGroom{\MYname{}}. If you have the room key with badge number 101, you may enter the room.}
  \s\MYitems {\iBible}
  } 
  
\NEW{Sign}{\sScientistRoom}{
  \s\MYname	{\cScientist{\MYname{}}'s Hotel Room}
  \s\MYloc	{- 004}
  \s\MYtext	{The name plaque says \cScientist{\MYname{}}. Below it is another name plaque that says \cKid{\Myname{}}. If you have the room key with badge number 102 or 103, you may enter the room.}
  \s\MYitems {\iBible}
  }
  
\NEW{Sign}{\sRivalRoom}{
  \s\MYname	{\cRival{\MYname{}}'s Hotel Room}
  \s\MYloc	{- 004}
  \s\MYtext	{The name plaque says \cRival{\MYname{}}.  If you have the room key with badge number 104, you may enter the room.}
  \s\MYitems {\iBible}
  }
  
\NEW{Sign}{\sPastorRoom}{
  \s\MYname	{\cRival{\MYname{}}'s Hotel Room}
  \s\MYloc	{- 004}
  \s\MYtext	{The name plaque says \cRival{\MYname{}}.  If you have the room key with badge number 105, you may enter the room.}
  \s\MYitems {\iBible}
  }
 
%% The Kitchen
\NEW{Sign}{\sMortarAndPestle}{
  \s\MYname	{A Mortar and Pestle}
  \s\MYloc	{- 002}
  \s\MYtext	{A Mortar and Pestle are lying on the counter. You may not interact with this sign unless you know otherwise.}
  \s\MYitems {\iPoisonedWine{}}
  }
  
\NEW{Sign}{\sRefrigerator}{
  \s\MYname	{Refrigerator}
  \s\MYloc	{- 002}
  \s\MYtext	{A big, industrial sized refrigerator. If you would like to search the refrigerator, place both hands on this sign for 30 seconds. Note: The items available here are effectively infinate in quantity. If this envelope is ever empty, tell a GM.}
  \s\MYitems {\multi{5}{\iMagnet{}} %% How do I get multiple magnets? is there a better way?
  }

\NEW{Sign}{\sKitchenSink}{
  \s\MYname	{The Kitchen Sink}
  \s\MYloc	{- 002}
  \s\MYtext	{A big, stainless steel sink. A good place to wash your hands. If you would like to search the sink, place both hands on this sign for 30 seconds.Note: The items available here are effectively infinate in quantity. If this envelope is ever empty, tell a GM.}
  \s\MYitems {\multi{5}{\iSteelWool{}} %% How do I get multiple magnets? is there a better way?
  }
  
\NEW{Sign}{\sKitchenCabinet}{
  \s\MYname	{A Kitchen Cabinet}
  \s\MYloc	{- 002}
  \s\MYtext	{Full of all kinds of useful cooking implements. If you would like to search the sink, place both hands on this sign for 30 seconds.Note: The items available here are effectively infinate in quantity. If this envelope is ever empty, tell a GM.}
  \s\MYitems {\multi{5}{\iPastaStrainer{}} %% How do I get multiple magnets? is there a better way?
  }
  
%% The Bar

\NEW{Sign}{\sBarCounter}{
  \s\MYname	{The Bar}
  \s\MYloc	{- 002}
  \s\MYtext	{Normally there would be a bartender here, mixing drinks. If you would like to search behind the counter, place both hands on this sign for 30 seconds.}
  \s\MYitems {\multi{10}{\iWine{}\multi{5}\iVial{}} %% How do I get multiple magnets? is there a better way?
  }
  
%% NEED THE TV and to associate the items (which haven't been built yet)
