%%%%
%%
%% This file sets up the Sign and Label datatypes and creates Sign and
%% Label macros.
%%
%% Signs generally represent interesting parts of game area, usually
%% as things posted on walls.  Labels represent other things, often on
%% or inside envelopes, that are part of complex mechanics.
%%
%% The default value for \MYloc will inherit location from the Place
%% or Sign most immediately up the ownership tree.  Override this by
%% setting \MYloc to anything (even blank).
%%
%% Sign is for full-sized signs that would cover most of a large
%% manila envelope; SignMedium is for signs sized to half-sized manila
%% envelopes; SignSmall is for signs sized for small manila envelopes
%% (the same size as item cards).  Label, LabelMedium, and LabelSmall
%% are analagous, but they don't have a \takedownby note at the
%% bottom.  You can always use a sign or label without an envelope or
%% with a differently-sized envelope.  Choose which based on
%% visibility and content.
%%
%% SignTiny is for signs you want to be hard to find; it is small and
%% does not have a \takedownby note.  SignDot is for a very small
%% "dot" which only has a title.
%%
%% SignStrip produces a strip of paper (without a \takedownby note)
%% with labels on the outside that show on both sides if you fold it
%% in half.  These are a convenient alternative to sub-envelopes. They
%% can also be used for "s-packets" taped to walls (see
%% Extras/README-s-packets).
%%
%% LabelCover produces a label similar to the cover to a research
%% notebook.  LabelPage, likewise, produces a page.
%%
%% EOG is for full-sized end-of-game signs.
%%
%%%%%

\DECLARESUBTYPE{Sign}{Element}
\PRESETS{Sign}{
  \FD\MYloc	{\mylocation} %% real-space location
  \FD\MYtext	{} %% text of sign
  }
\POSTSETS{Sign}{
  \edef\mylocation{\MYloc}
  \protected@edef\@ownerstring{%
    \MYname%
    \ifx\mylocation\empty\else\ (\mylocation)\fi%
    }
  }
\def\mylocation{}

\def\loc#1{\rs\MYloc{#1}}

\DECLARESUBTYPE{SignMedium}{Sign}
\DECLARESUBTYPE{SignSmall}{Sign}
\DECLARESUBTYPE{SignTiny}{Sign}
\DECLARESUBTYPE{SignDot}{Sign}
\PRESETS{SignDot}{\s\MYtext{}}

\DECLARESUBTYPE{Label}{Sign}
\PRESETS{Label}{\s\MYloc{}}
\DECLARESUBTYPE{LabelMedium}{Label}
\DECLARESUBTYPE{LabelSmall}{Label}

\DECLARESUBTYPE{SignStrip}{Sign}
\DECLARESUBTYPE{LabelCover}{Label}
\DECLARESUBTYPE{LabelPage}{Label}

\DECLARESUBTYPE{EOG}{Sign}
\PRESETS{EOG}{%
  \s\MYname	{End Of Game}
  \s\MYtext	{{\bf\Huge You may not pass through here.}}
  }


%%%%%
%% \signbig[<location>]{<name>}{<text>}
%% \eog[<location>]
%%
%% \signmdeium[<location>]{<name>}{<text>}
%% \signsmall[<location>]{<name>}{<text>}
%% \signtiny[<location>]{<name>}{<text>}
%% \signdot[<location>]{<name>}
%%
%% \labelbig{<name>}{<text>}
%% \labelmedium{<name>}{<text>}
%% \labelsmall{<name>}{<text>}
%%
%% \signstrip[<location>]{<name>}{<text>}
%% \labelcover{<name>}{<text>}
%% \labelpage{<name>}{<text>}
\newinstance{Sign}{\signbig[3][\mylocation]}{
  \s\MYloc{#1}\s\MYname{#2}\s\MYtext{#3}}
\newinstance{EOG}{\eog[1][\mylocation]}{\s\MYloc{#1}}

\newinstance{SignMedium}{\signmedium[3][\mylocation]}{
  \s\MYloc{#1}\s\MYname{#2}\s\MYtext{#3}}
\newinstance{SignSmall}{\signsmall[3][\mylocation]}{
  \s\MYloc{#1}\s\MYname{#2}\s\MYtext{#3}}
\newinstance{SignTiny}{\signtiny[3][\mylocation]}{
  \s\MYloc{#1}\s\MYname{#2}\s\MYtext{#3}}
\newinstance{SignDot}{\signdot[2][\mylocation]}{
  \s\MYloc{#1}\s\MYname{#2}}

\newinstance{Label}{\labelbig[2]}{
  \s\MYname{#1}\s\MYtext{#2}}
\newinstance{LabelMedium}{\labelmedium[2]}{
  \s\MYname{#1}\s\MYtext{#2}}
\newinstance{LabelSmall}{\labelsmall[2]}{
  \s\MYname{#1}\s\MYtext{#2}}

\newinstance{SignStrip}{\signstrip[3][\mylocation]}{
  \s\MYloc{#1}\s\MYname{#2}\s\MYtext{#3}}
\newinstance{LabelCover}{\labelcover[2]}{
  \s\MYname{#1}\s\MYtext{#2}}
\newinstance{LabelPage}{\labelpage[2]}{
  \s\MYname{#1}\s\MYtext{#2}}


%%%%%
%% \sEOG{}
%% use \sEOg[\loc{<location>}]{} for EOG sign at a specific place
\NEW{EOG}{\sEOG}{
  }


%%%%%%%%%%%%%%%%%%%%%%%%%%%%%%%%%%%%%%%%%%%%%%%%%%%%%%%%%%%%%%%%%%

\NEW{Sign}{\sTest}{
  \s\MYname	{A Room}
  \s\MYloc	{10-250}
  \s\MYtext	{A lecture hall with large, sliding blackboards.}
  }


%%%%%%%%%%%%%%%%%%%%%%%%%%%%%%%%%%%%%%%%%%%%%%%%%%%%%%%%%%%%%%%%%%

%% The hotel rooms
\NEW{Sign}{\sHotelRooms}{
  \s\MYname	{Hotel Room Access}
  \s\MYloc	{- 004}
  \s\MYtext	{There are two ways to gain entrance to a hotel room unless you know otherwise:

{\bf 1. } Possess the appropriate room key.

{\bf 2. } You may hit the door with a {\bf CR of 5} or greater.

Once you have completed one of the above requirements, you may stand with your hand on the room sign for 30 seconds to enter the room. You may then freely stash items that are stashable, and take items that are in the room.}
  }

\NEW{Sign}{\sBrideRoom}{
  \s\MYname	{A Hotel Room}
  \s\MYloc	{- 004}
  \s\MYtext	{The nameplaque says {\bf \cBride{\MYname{}}}. If you have the room key with item {\bf \# 100}, you may enter the room.}
  \s\MYitems {\iBible{}}
  }
  
\NEW{Sign}{\sGroomRoom}{
  \s\MYname	{A Hotel Room}
  \s\MYloc	{- 004}
  \s\MYtext	{The name plaque says {\bf \cGroomA{\MYname{}}}. If you have the room key with item {\bf \# 101}, you may enter the room.}
  \s\MYitems {\iBible{}}
  } 
  
\NEW{Sign}{\sScientistRoom}{
  \s\MYname	{A Hotel Room}
  \s\MYloc	{- 004}
  \s\MYtext	{The name plaque says {\bf \cScientist{\MYname{}}}. Below it is another name plaque that says {\bf \cKid{\MYname{}}}. If you have the room key with item {\bf \# 102 or 103}, you may enter the room.}
  \s\MYitems {\iBible{}\iBall{}}
  }
  
\NEW{Sign}{\sRivalRoom}{
  \s\MYname	{A Hotel Room}
  \s\MYloc	{- 004}
  \s\MYtext	{The name plaque says {\bf \cRival{\MYname{}}}.  If you have the room key with item {\bf \# 104}, you may enter the room.}
  \s\MYitems {\iBible{}}
  }
  
\NEW{Sign}{\sPastorRoom}{
  \s\MYname	{A Hotel Room}
  \s\MYloc	{- 004}
  \s\MYtext	{The name plaque says {\bf \cPastor{\MYname{}}}.  If you have the room key with item {\bf \# 105}, you may enter the room.}
  \s\MYitems {\iBible{}}
  }
 
%% The Kitchen
\NEW{Sign}{\sMortarAndPestle}{
  \s\MYname	{A Mortar and Pestle}
  \s\MYloc	{- 002}
  \s\MYtext	{A Mortar and Pestle are lying on the counter. You may not interact with this sign unless you know otherwise.}
  \s\MYitems {\iPoisonedWine{}}
  }
  
\NEW{Sign}{\sRefrigerator}{
  \s\MYname	{Refrigerator}
  \s\MYloc	{- 002}
  \s\MYtext	{A big, industrial sized refrigerator. If you would like to search the refrigerator, place both hands on this sign for 30 seconds. Note: The items available here are effectively infinate in quantity. If this envelope is ever empty, tell a GM.}
  \s\MYitems {\multi{5}{\iMagnet{}}} 
  }

\NEW{Sign}{\sKitchenSink}{
  \s\MYname	{The Kitchen Sink}
  \s\MYloc	{- 002}
  \s\MYtext	{A big, stainless steel sink. A good place to wash your hands. If you would like to search the sink, place both hands on this sign for 30 seconds. Note: The items available here are effectively infinate in quantity. If this envelope is ever empty, tell a GM.}
  \s\MYitems {\multi{5}{\iSteelWool{}}} 
  }
  
\NEW{Sign}{\sKitchenCabinet}{
  \s\MYname	{A Kitchen Cabinet}
  \s\MYloc	{- 002}
  \s\MYtext	{Full of all kinds of useful cooking implements. If you would like to search the cabinet, place both hands on this sign for 30 seconds. }
  \s\MYitems {\multi{3}{\iPastaStrainer{}} \multi{3}{\iKnife{}}} 
  }
  
%% The Bar

\NEW{Sign}{\sBarCounter}{
  \s\MYname	{The Bar}
  \s\MYloc	{- 002}
  \s\MYtext	{Normally there would be a bartender here, mixing drinks. If you would like to search behind the counter, place both hands on this sign for 30 seconds. You may then take 1 item from the counter.}
  \s\MYitems {\multi{10}{\iWine{}}\multi{7}{\iVial{}}} 
  }
  
%% NEED THE TV and to associate the items [solenoid, whatever else, see the escape mechanic] (which haven't been built yet)

%%The construction site
\NEW{Sign}{\sToolbox}{
  \s\MYname	{A Toolbox}
  \s\MYloc	{Hall off of - 002}
  \s\MYtext	{If you would like to search the toolbox, place a hand on this sign for 1 minute. You may then take 1 item of your choosing from the toolbox.}
  \s\MYitems {\multi{3}{\iSolderingIron{}} }
  }
  
\NEW{Sign}{\sPileOfWood}{
  \s\MYname	{A Pile of Wood}
  \s\MYloc	{Hall off of - 002}
  \s\MYtext	{An assorted collection of wooden planks and boards. If you would like to search the pile, place a hand on this sign for 1 minute. You may then take 1 item of your choosing from the pile.}
  \s\MYitems {\multi{3}{\iTwoByFour{}} }
  }
  
%% The Lobby of the Hotel
\NEW{Sign}{\sReception}{
  \s\MYname	{The Reception desk}
  \s\MYloc	{Main Hallway}
  \s\MYtext	{Normally there would be someone staffing the desk.}
  }
  
\NEW{Sign}{\sPortrait}{
  \s\MYname	{A Picture of the God of Virtuism}
  \s\MYloc	{Main Hallway}
  \s\MYtext	{A beautiful rendering of the image of the God of Virtuism.}
  }

%%The flower pot anomoly!

%%REALITY
\NEW{Sign}{\sHallwayBlocked}{
  \s\MYname	{YOU MAY NOT PASS THIS SIGN UNLESS YOU KNOW OTHERWISE}
  \s\MYloc	{-015}
  \s\MYtext	{\#9999}
  }

\NEW{Sign}{\sPossessABody}{
  \s\MYname	{How to Possess A Body}
  \s\MYloc	{-015}
  \s\MYtext	{If your badge number is anything other than {\bf \cPassive{\s\MYnumber{}}} or {\bf \cActive{\MYnumber{}}}, take one of the index cards below to represent your robot body while you are here. If there are no more index cards available, there is no body for your mind to transfer into, and you are returned to the simulation. (You fail to gain access to this room. Return to game space beyond the blue tape barrier).} %% We will tape 3 notecards to the door below this sign
  }

\NEW{Sign}{\sAlienBodies}{
  \s\MYname	{A Pair of alien bodies}
  \s\MYloc	{-015}
  \s\MYtext	{There are two bodies seated cross legged on the floor, with their eyes closed as if asleep. You have never seen bodies like these before. They must be aliens.
  
  If there is a notecard below this sign, you may attempt to kill one of the bodies if you choose. If you do so, you must inform a GM before you begin. The action takes 5 minutes, and is interruptable. If you succeed in killing the body, tear up one of the notecards. If there are no notecards, both bodies have been killed.}  %% We will tape 2 notecards to the wall below this sign
  }

\NEW{Sign}{\sReturnToSim}{
  \s\MYname	{How to return to simulation}
  \s\MYloc	{-015}
  \s\MYtext	{If at any time you wish to return to the simulation, simply replace the index card representing your body where you found it, set your badge to ``I'm not here'' and return to the deck of cards. Then set your badge back.}
  }

\NEW{Sign}{\sSimStateExplained}{
  \s\MYname	{The Simulation State Readout}
  \s\MYloc	{-015}
  \s\MYtext	{The signs below indicate the state of the simulation.  If either one does not read {\bf ``Broken''}, then the simulation has been altered from the broken state. If one of the signs reads otherwise, then the state of the simulation is such.
  
  If you would like to alter the state of the simulation, follow the instructions on one of the signs below.}
  }

\NEW{Sign}{\sBrokenSim}{
  \s\MYname	{The Simulation is {\bf Broken}}
  \s\MYloc	{-015}
  \s\MYtext	{The simulation has experienced a fatal error. At 11:00 pm, the simulation will be terminated. All data will be lost. REPEAT: All data will be lost.
  
  If you have item {\bf \#126}, and you place it in the blue square below (remove anything already in the blue square), replace this sign with the one under it. If the sign to the right does not indicate that the simulation is {\bf broken}, swap the signs so it does.}
  }

\NEW{Sign}{\sBrokenSimTwo}{
  \s\MYname	{The Simulation is {\bf Broken}}
  \s\MYloc	{-015}
  \s\MYtext	{The simulation has experienced a fatal error. At 11:00 pm, the simulation will be terminated. All data will be lost. REPEAT: All data will be lost.
  
  If you have item {\bf \#148}, and you place it in the blue square below (remove anything already in the blue square), replace this sign with the one under it. If the sign to the left does not indicate that the simulation is {\bf broken}, swap the signs so it does.}
  }
  
\NEW{Sign}{\sFixedSim}{
  \s\MYname	{The {\bf Save State} is present}
  \s\MYloc	{-015}
  \s\MYtext	{The fatal errors have been corrected. At 11:00 pm, the simuation will revert to the save state.  All anomolies will be removed.}
  }

\NEW{Sign}{\sUtopianSim}{
  \s\MYname	{The {\bf Utopia State} is present}
  \s\MYloc	{-015}
  \s\MYtext	{The simulation has been modified to reflect a utopian state. At 11:00 pm, the changes will take effect.}
  }

\NEW{Sign}{\sRope}{
  \s\MYname	{A coil of rope}
  \s\MYloc	{-015}
  \s\MYtext	{There is a long coil of rope in this corner. It can't be moved very far, but two players may use it to restrain a third. The two players wishing to use the rope must each put one hand on this sheet and then indicate the player to be restrained. The restrained player must comply, but is immediately released should either restraining player remove their hand from this sheet. This item has a 2 minute cool down time while you untangle the rope after a player is released (intentionally or accidentally).}
  }         
  

%%The Gift Shop
\NEW{Sign}{\sGiftShop}{
  \s\MYname	{The Gift Shop}
  \s\MYloc	{-030}
  \s\MYtext	{This is the gift shop for the hotel. It is small and mostly full of cheap junk.}
  \s\MYitems {}
  }

\NEW{Sign}{\sGiftShopDisplay}{
  \s\MYname	{A Display in the Gift Shop}
  \s\MYloc	{-030}
  \s\MYtext	{There are many things on display. If you would like to take one, place your hand on the sign for 1 minute, then you may take an item of your choice.}
  \s\MYitems {\iCrayons{}\iCrayons{}\iHairbrush{}\iHairbrush{}\multi{4}{\iShoePolish{}}}
  }
  
\NEW{Sign}{\sGiftShopStuffies}{
  \s\MYname	{A Display of stuffed animals}
  \s\MYloc	{-030}
  \s\MYtext	{There are a whole host of stuffed animals on display here. If you would like to search the pile of stuffed animals, place your hand on the sign for 1 minute, then you may take an item if one is available.}
  \s\MYitems {\iBunnyRabbit{}}%%A GM will put the item into the envelope when the utopia state appears in game
  }
  
%% The High Security Door
\NEW{Sign}{\sHighSecurityDoorLocked}{
  \s\MYname	{A High Security Door}
  \s\MYloc	{Near -013}
  \s\MYtext	{This steel door looks important. It has a keyhole instead of a card reader. If you have item {\bf \# 125} you may open the door. Replace this sign with the one under it (put this sign under that one). You may then take any item from this room that you wish to take.}
  \s\MYitems {\iMilitarySecret{}}
  }

\NEW{Sign}{\sHighSecurityDoorUnlocked}{
  \s\MYname	{A High Security Door}
  \s\MYloc	{Near -013}
  \s\MYtext	{This steel door looks important. It has been left open. You may close the door. If you do so, replace this sign with the one under it (put this sign under that one).}
  \s\MYitems {}
  }

%% Broom Closet
\NEW{Sign}{\sBroomCloset}{
  \s\MYname	{A Broom Closet full of junk}
  \s\MYloc	{-032}
  \s\MYtext	{This is a small broom closet. There seems to be a whole bunch of random items in here if you had the patience to find them.}
  \s\MYitems {}
  }
  
\NEW{Sign}{\sPileOfJunk}{
  \s\MYname	{A Pile of Junk}
  \s\MYloc	{-032}
  \s\MYtext	{There is a pile of junk here. If you would like search the pile of junk for a useful item, spend 3 minutes searching this room. You may then take one item randomly.}
  \s\MYitems {}%%need the widget in here.
  }
  
%% The ballroom
\NEW{Sign}{\sBallroom}{
  \s\MYname	{A Pile of Junk}
  \s\MYloc	{-032}
  \s\MYtext	{There is a pile of junk here. If you would like search the pile of junk for a useful item, spend 3 minutes searching this room. You may then take one item randomly.}
  \s\MYitems {}%%need the widget in here.
  }