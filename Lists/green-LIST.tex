%%%%%
%%
%% This file sets up the Green datatype and creates Green macros.
%% These are for greensheets.
%%
%%%%%

\DECLARESUBTYPE{Green}{Element}
\PRESETS{Green}{
  \FD\MYfile	{} %% filename (in \greens dir)
  \FS\MYtext	{\ifx\MYfile\empty\else%
		  \getextractenvs{document}{\greens/\MYfile}%
		\fi}
  }


%%%%%%%%%%%%%%%%%%%%%%%%%%%%%%%%%%%%%%%%%%%%%%%%%%%%%%%%%%%%%%%%%%

\NEW{Green}{\gTest}{
  \s\MYname	{Test Greensheet}
  \s\MYfile	{README.tex}
  }


%%%%%%%%%%%%%%%%%%%%%%%%%%%%%%%%%%%%%%%%%%%%%%%%%%%%%%%%%%%%%%%%%%

\NEW{Green}{\gPrepareWedding}{
  \s\MYname	{Wedding Preparations}
  \s\MYfile	{weddingprep.tex}
  }
  
  \NEW{Green}{\gWedding}{
  \s\MYname	{The Wedding Ceremony}
  \s\MYfile	{wedding.tex}
  }

\NEW{Green}{\gVirus}{
  \s\MYname	{$\Omega$}
  \s\MYfile	{virus.tex}
  }
  
  %%%%%%%%%%%%%%%%%%%%%%%%%%%%%%%%%%%%%%%%%%%%%%%%%%%%%%%%%%%%%%%%%%%%%