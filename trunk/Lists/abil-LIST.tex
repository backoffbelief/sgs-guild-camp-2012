%%%%%
%%
%% This file sets up the Abil datatype and creates Abil macros.  These
%% are for abilities that characters may have.
%%
%%%%%

\DECLARESUBTYPE{Abil}{Element}
\PRESETS{Abil}{
  \F\MYtext	%% text of ability, read by user
  \F\MYeffect	%% effect text of ability, read by recipient(s)
  }


%%%%%
%% \ability{<name>}{<text>}{<effect>}
%%
%% \ability is a wrapper around \INSTANCE, useful for 1-shot abilities,
%% etc.
\newinstance{Abil}{\ability[3]}{
  \s\MYname{#1}\s\MYtext{#2}\s\MYeffect{#3}}


%%%%%%%%%%%%%%%%%%%%%%%%%%%%%%%%%%%%%%%%%%%%%%%%%%%%%%%%%%%%%%%%%%

\NEW{Abil}{\aTest}{
  \s\MYname	{Test Ability}
  \s\MYtext	{You are a test.}
  \s\MYeffect	{This is a Test.}
  }

\NEW{Abil}{\aSpecial}{
  \s\MYname	{Special Powers}
  \s\MYtext	{You have special powers, as detailed in your \gTest{}
		greensheet.}
  \s\MYeffect	{I have special powers!}
  \s\MYgreens	{\gTest{}}
  \suite
  }

\NEW{Abil}{\aFiremansCarry}{
  \s\MYname	{Fireman's Carry}
  \s\MYtext	{You can carry a body as if it were two hands bulky.}
  \s\MYeffect	{I can carry this body well.}
  }


%% Basic DarkWater-style Martial Attack abilities


%% Everyone has these 3
\NEW{Abil}{\aAssist}{
  \s\MYname	{Assist}
  \s\MYtext	{You can assist someone else's attack.  You must be
		within ZoC of both the attacker and target.  Within two
		seconds of an attack, direct this at the attacker,
		saying ``\MYname'' and your CR.}
  \s\MYeffect	{I assist your attack.}
  }

\NEW{Abil}{\aKnockOut}{
  \s\MYname	{Knock Out}
  \s\MYtext	{You can knock someone out as an attack.  This requires
		a {\bf blunt} weapon.  Say ``\MYname'' and your CR.}
  \s\MYeffect	{I knock you out.}
  }

\NEW{Abil}{\aWound}{
  \s\MYname	{Wound}
  \s\MYtext	{You can wound someone as an attack.  This requires an
		{\bf edged} weapon, such as a knife.  Say ``\MYname'' and
		your CR.}
  \s\MYeffect	{I wound you.}
  }


%% the \basecombat macro can be prepended to the Char abils list
%% (in char-LIST.tex)
\def\basecombat{\aKnockOut{}\aWound{}\aAssist{}}


%% only some people have these
\NEW{Abil}{\aDisarm}{
  \s\MYname	{Disarm}
  \s\MYtext	{You can disarm one item from someone as an attack.  Say
		``\MYname'' and your CR.  Point at the item you want to
		disarm.  If the attack works, they must drop that item.}
  \s\MYeffect	{I disarm that item.}
  }

\NEW{Abil}{\aRestrain}{
  \s\MYname	{Restrain}
  \s\MYtext	{You can restrain someone as an attack.  Say ``\MYname''
		and your CR.  You may freely drag, attack, or (if you have
		a weapon) killing-blow them.  To do anything else, or if
		your health state changes, incant ``release'' and let them
		go.}
  \s\MYeffect	{I restrain you.  You are restrained until I incant
		``release.''}
  }

\NEW{Abil}{\aThrow}{
  \s\MYname	{Throw}
  \s\MYtext	{You can throw someone as an attack.  Say ``\MYname'' and
		your CR.  Point in the direction you want to throw them.}
  \s\MYeffect	{I throw you.  Go in the direction I point ten full steps
		or until you hit a wall or similar.}
  }
  
  \NEW{Abil}{\aPickPocket}{
  \s\MYname	{Pick Pocket}
  \s\MYtext {You can pick pockets.  You have a sheet of {\bf stickers}. If you can 
  place a sticker undetected on the person you want to steal from, find a GM.  
  You can steal one random item or attempt to steal one specific item. For a specific 
  item, give the GM a description of the item you want to steal and the GM will 
  determine if your grab is successful and if it is, take the item from your victim 
  and give it to you.  If they do not have that item, you get nothing. For a random 
  item, tell the GM you want to steal a random item and the GM will take one 
  item from your victim and give it to you. You may also use this ability to place 
  an item in someone's pocket.  In this case, give the item you want to place to 
  a GM. If you are caught trying to place the sticker, you must tell your victim that 
  they caught you attempting to pickpocket them. {\bf NOTE:} you may not 
  pickpocket Bulky items}
  \s\MYeffect	{I can pick pockets.}
  }


%%%%%%%%%%%%%%%%%%%%%%%%%%%%%%%%%%%%%%%%%%%%%%%%%%%%%%%%%%%%%%%%%%
