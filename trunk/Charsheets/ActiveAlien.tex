\documentclass[char]{guildcamp1}
\begin{document}
\name{\cActive{}}

%Passive's name is: Quarth
%Active's name is: Krung

Your name is Krung. You are a scout for the military wing of a galaxy-spanning species that calls themselves the \emph{Xenids}. Your species started off as an eight-legged race that relied upon bulky claws for fine-scale manipulation. Over the centuries your species shed its apprehension to large-scale genetic engineering and adopted a bipedial, two-armed form that you copied from several other successful species. Your civilization is very powerful, but is currently at war with a smaller but more advanced galaxy-spanning civilization called the \emph{Illithids}. You are losing the war and some of the planets on the border stars of your civilization are being conquered --- including your home planet. Although the core stars of the Xenid civilization are still quite safe, You fully intend to reclaim your occupied home planet one day.

The majority of the Xenids, although not strictly xenophobic, prefer to stay on Xenid-controlled planets and do not interact with other species. Although you would prefer to remain among your own species, you realize that to overcome the Illithids you will need the help of species more advanced than your own. Towards this end you joined \emph{Contact}, the exploratory and scouting branch of the Xenid hierarchy. Contact's primary goal is to gather data on new species: ``How technologically advanced are they?''; ``How does their government work?''; ``Are they a military threat?''; ``Are they a potential ally?''. There are hundreds of billions of stars in the Milky Way and only some support life, so there are always plenty of uncharted stars out there.

Your current contact assignment is to investigate an unknown energy signature detected at a planetless blue dwarf star. You are in a two-person vessel, and your partner is Quarth. She is a scholar from the Xenid home world. Although very intelligent, she sometimes seems to forget your civilization is losing an interstellar war and, when encountering new species, focuses too much on understanding the minutia and misses out on the bigger picture. She is also overly concerned with the ``rules'' of first-contact: these are noble ideas in peace time, but while at war some risks have to be taken to ensure the survival of the species.

On your way to the blue dwarf, your sensors picked up what you believed to be an unknown ship traveling at a very high fraction of the speed of light. You and Quarth debated for some time what action to take. You argued that the potential benefits outweighed the risks; a cursory material analysis suggests that whoever built this ship is technologically comparable to, and might even surpass, the \emph{Illithids}. Quarth argued that the unknown ship was not emitting any signals and was clearly not interested in being discovered, so might destroy your small scout ship to maintain its stealth profile. While you were busy debating this, the ship's main power supply seemed to falter and die, and it started emitting a string of sequential binary-coded prime numbers. Quarth took the sequence of primes to be a distress signal of sorts, and finally agreed to get closer and investigate the ship.

The ship consists of four main components: a solar array, a battery system, an ion engine, and a massive computer. The solar array and battery system are superior to anything your civilization can manufacture, with an almost 100\% energy conversion ratio and almost no passive battery loss, while the ion engine is mediocre at best. The computer is the part that fascinates you the most, as it clearly surpasses both Xenid and Illithid design. It consists of a wide array of subatmoically packed memory cells tightly intermixed with equally small-scale processing units. There is no atmosphere on the ship, but there are three robot bodies wirelessly hooked up to the computer system.

After some analysis, you determine that the computer seems to be designed to run a giant simulation. The odd powering down of the ship's system you witnessed recently seems to have caused all but a very small part of the computer system to shut down, but fortunately has also opened up a simple debug layer to the simulation. This layer contains a rather lengthy tutorial clearly designed for contact with other species that starts with basic mathematics and works forward into the basics of science, culture, and language. The species that built this system refers to itself as ``human'', and if this tutorial can be trusted, this simulation and the ship that houses it, called the \emph{New Eden}, is most of what remains of their civilization.

For your own reasons, you and Quarth are both eager to learn more about these humans. Unfortunately the information contained in the tutorial is not nearly sufficient. If you want to learn more, your only opportunity seems to be to try and interface with the simulation directly. Fortunately, the newly exposed debug layer and corresponding tutorial provides a simple way to do this; these humans seem to have been masters of the mind-machine interface, which only reinforces how important it is to learn more about them. Before you dive into the simulation, you take a moment to learn all that you can from within the ship.

You are not sure what caused this mysterious power-down of the ships systems, but you do know that there are only four hours of battery life remaining on the functioning part of the simulation.
Based on the ship's design, you are confident that these humans could tip the balance of the war in your favor. Towards that end you will go to any lengths to make sure that you secure the aid your race needs, either by convincing the humans to help you directly in your war or providing you with the technological superiority you need.

With a rudimentary knowledge of human language, the two of you wirelessly interface with the computer simulation. The first thing you notice is that the simulation is totally immersive and an effectively perfect recreation of reality: you are not sure exactly how to disconnect yourself from the simulation and return to your body. The tutorial claimed this would be easy, but the ``escape console'' it said would materialize in front of you upon connecting is nowhere to be found. You seem to have been put in control of one of these fleshy human bodies, what they refer to as a female. The tutorial was not very specific on what the phrase ``gender'' actually means or what the different between the two sexes are. You make a mental note to investigate that further.

You take a look at your surroundings. Your genetically-perfected memory comes in handy here, as you start mapping pictures to vocabulary. You are in a room with one window, there is a bed, a lamp, a small desk, a closet, and a small side room containing a toilet, a shower, and a sink. You have a key in your pocket with a small magnetic strip. After some looking around you conclude that you are in what is called a ``hotel room''. Distressingly, you are alone and Quarth is nowhere to be found. The technology level of whoever built this room also appears to be absurdly primitive compared to the civilization that built the ship currently housing your body. This deeply concerns you --- whoever built this ``hotel'' can probably not even manage to go into space, much less help you in your war. Nevertheless, you are determined to find someone who knows what is going on and who can help you in your quest. Why are the humans that built this ship so advanced while those in this simulation so primitive?

Even if you cannot find someone to teach you human technology, your leaders can benefit from interacting with even these primitive humans to gain a better understanding of the military tactics and ideas of other species. Try to bring as many willing participants back with you; your analysis of the computer system suggests these simulated humans should be able to inhabit the robot bodies on the ship.

The biological composition of new species can also prove quite valuable. Your species often incorporates superlative features from other species into its own genetic makeup, and in extreme situations you can use genetic knowledge about a species to develop biological weapons to target similar species. Towards that end, you want to collect blood samples from several humans; the tutorial you interfaced with suggests that blood contains the entire genetic blueprint for the species, which you find a most exciting property.

The disconnection of your consciousness from your body poses an additional problem. Your species does not require sleep, but periodically your brain requires conscious maintenance to the memory network that you cannot provide. You are not sure what the consequences will be within the simulation, but you would like to find a way to avoid it.

\begin{itemz}[Goals]
  \item Find Quarth in the simulation. She's likely stuck in a body similar to yours.
  \item Your overriding reason for entered the simulation is to find allies or military technology to help your species.
  \item Convince humans to leave the simulation with you and return to the Xenid homeworld.
  \item Collect blood samples from up to five different humans.
  \item Find a way to prevent the mental degradation of your body back in the human spaceship.
\end{itemz}

%''Collect blood sample'' ability.
  %"if you spend 1 minute with a syringe item, a vial item, and a willing or helpless character, take a blank item sheet and write "X's blood" on it"

%The aliens must experience the full range of human emotions to cure their mental fog.  
%''free your mind'' mempacket.
% Love: Hold hands with three different humans for at least 5 seconds.
% Embarassment: Spin in circles for 10 seconds with another human watching.
% Contemplation: Go meditate in a room by yourself for one minute. ``What is the meaning of life?'' What frustrating thoughts!
% Spirituality: Receive a blessing from a religious figure or participate in a religious ceremony.
  
%%%%%
%% List contacts, using \contact{<char macro>}
\begin{contacts}
  \contact{\cPassive{}}: A fellow Xenid and your primary partner within Contact.
\end{contacts}

\end{document}
